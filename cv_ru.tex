\documentclass{moderncv}
\moderncvtheme[blue]{classic}
\usepackage[utf8]{inputenc}
\usepackage[T2A]{fontenc}
\usepackage[russian]{babel}
\usepackage{url}
\usepackage[scale=0.85]{geometry}
\recomputelengths

% Personal info
\firstname{Александр}
\familyname{Половцев}
\title{Программист}
\phone[mobile]{+7 (921) 340-55-39}
\email{alex.polovtcev@gmail.com}
\photo[80pt]{photo.jpg}
\social[github]{sashapolo}
\social[linkedin]{aleksandr-polovtsev}

\begin{document}
\makecvtitle

% Work experience
\section{Опыт работы}

\cventry{May 2015 -- Sep 2020}
        {Senior Java developer}
        {Genestack}
        {Санкт-Петербург}
        {Россия}
        {Разработка data management платформы для биологических данных. \newline{} Обязанности и достижения:
        \begin{itemize}
            \item Разработка сервиса для поиска и отображения информации об исследованиях. 
            Сервис превосходил раннее созданный прототип по производительности и впоследствии 
            позиционировался как одна из основных функциональностей продукта;
            \item Занимался техническим руководством небольшой команды по интеграции со сторонними системами,
            проект завершился успешным запуском на стороне заказчика в установленный срок;
            \item Разработка прототипа индекса для данных мутаций. Полученное решение на момент написания показало
            лучшую производительность и настраиваемость в сравнении с коробочными продуктами;
            \item Исполнял некоторые функции тим лида: написание дизайн документов с последующим разбиением на задачи для других разработчиков, код ревью, проведение one-on-one встреч; 
            \item Участвовал в проведении собеседованиий.
        \end{itemize}}

\cventry{Oct 2012 -- May 2015}
        {Full stack developer}
        {Санкт-Петербургский Политехнический Университет}
        {Санкт-Петербург}
        {Россия}
        {\begin{itemize}
            \item Разработка информационной системы для нужд университета с использованием фреймворка Django;
            \item Разработка программы по динамическому анализу многопоточных программ, основанному на визуализации
            потоков исполнения, с использованием Clang API;
            \item Разработка транслятора DSL для описания семантики функций.
        \end{itemize}}

% Job skills
\section{Профессиональные навыки}

\subsection{Языки программирования}
\cvitem{Хороший}{Java, Kotlin, Python, SQL}
\cvitem{Средний}{C++}
\cvitem{Начальный}{Bash}

\subsection{Фреймворки и другие технологии}
\cvitem{Фреймворки}{Spring, Spring Boot}
\cvitem{Базы данных}{MySQL, Apache Solr, ClickHouse}
\cvitem{Тестирование}{JUnit, Mockito, Testcontainers, Pytest}
\cvitem{Контейнеры}{Docker, Docker Compose}
\cvitem{VCS}{Git}
\cvitem{Системы сборки}{Maven, Gradle}

% Education
\section{Образование}

\cventry{2012 -- 2014}
        {Магистр техники и технологии}
        {Кафедра компьютерных систем и программных технологий, Санкт-Петербургский Политехнический Университет}
        {Санкт-Петербург, Россия}
        {\textit{диплом с отличием}}
        {}

\cventry{2008 -- 2012}
        {Бакалавр}
        {Кафедра компьютерных систем и программных технологий, Санкт-Петербургский Политехнический Университет}
        {Санкт-Петербург, Россия}
        {\textit{диплом с отличием}}
        {}

\section{Магистерская диссертация}

\cvitem{тема}{\emph{Инструментальная среда для анализа программных систем}}
\cvitem{описание}{Разработка системы по анализу и визуализации свойств программ на основе метамодели, независимой от 
                  языка программирования. Разработанный прототип поддерживал несколько видов визуализации структуры
                  (UML диаграммы, граф потока исполнения) и подсчет простых метрик для программ 
                  на языке C и Java.}

% Other skills and interests
\section{Языки}
\cvlanguage{Русский}{Родной}{}
\cvlanguage{Английский}{Свободное владение}{}
\cvlanguage{Французский}{Начинающий}{}

\section{Хобби}
\cvitem{Спорт}{Тяжелая атлетика, бег}
\cvitem{Обучение}{Прохождение онлайн курсов и получение новых знаний на сайтах типа \href{https://www.coursera.org/}{Coursera} или \href{https://www.udemy.com/}{Udemy}}

% Bibliography
\nocite{*}
\bibliographystyle{plain}
\renewcommand{\refname}{Публикации}
\bibliography{publications}

\end{document}
